% !TEX program = lualatex
%%%%%%%%%%%%%%%%%
% This is an example CV created using altacv.cls (v1.1.4, 27 July 2018) written by
% LianTze Lim (liantze@gmail.com), based on the
% Cv created by BusinessInsider at http://www.businessinsider.my/a-sample-resume-for-marissa-mayer-2016-7/?r=US&IR=T
%
%% It may be distributed and/or modified under the
%% conditions of the LaTeX Project Public License, either version 1.3
%% of this license or (at your option) any later version.
%% The latest version of this license is in
%%    http://www.latex-project.org/lppl.txt
%% and version 1.3 or later is part of all distributions of LaTeX
%% version 2003/12/01 or later.
%%%%%%%%%%%%%%%%

%% If you want to use \orcid or the
%% academicons icons, add "academicons"
%% to the \documentclass options.
%% Then compile with XeLaTeX or LuaLaTeX.
% \documentclass[10pt,a4paper,academicons]{altacv}

%% Use the "normalphoto" option if you want a normal photo instead of cropped to a circle
% \documentclass[10pt,a4paper,normalphoto]{altacv}

\documentclass[10pt,a4paper]{altacv}

%% AltaCV uses the fontawesome and academicon fonts
%% and packages.
%% See texdoc.net/pkg/fontawecome and http://texdoc.net/pkg/academicons for full list of symbols.
%% When using the "academicons" option,
%% Compile with LuaLaTeX for best results. If you
%% want to use XeLaTeX, you may need to install
%% Academicons.ttf in your operating system's font %% folder.


% Change the page layout if you need to
\geometry{left=1cm,right=9cm,marginparwidth=6.8cm,marginparsep=1.2cm,top=1cm,bottom=1cm}

% Change the font if you want to.

% If using pdflatex:
\usepackage[utf8]{inputenc}
\usepackage[T1]{fontenc}
\usepackage[default]{lato}
\usepackage{hyperref}

% If using xelatex or lualatex:
% \setmainfont{Lato}

% Change the colours if you want to
\definecolor{VividPurple}{HTML}{97BB00}
\definecolor{SlateGrey}{HTML}{2E2E2E}
\definecolor{LightGrey}{HTML}{666666}
\colorlet{heading}{VividPurple}
\colorlet{accent}{VividPurple}
\colorlet{emphasis}{SlateGrey}
\colorlet{body}{LightGrey}

% Change the bullets for itemize and rating marker
% for \cvskill if you want to
\renewcommand{\itemmarker}{{\small\textbullet}}
\renewcommand{\ratingmarker}{\faCircle}

%% sample.bib contains your publications
\addbibresource{sample.bib}

\begin{document}
\name{Lea Schlatterer}
\tagline{Social Arbaiterin}
% Cropped to square from https://en.wikipedia.org/wiki/Marissa_Mayer#/media/File:Marissa_Mayer_May_2014_(cropped).jpg, CC-BY 2.0
\photo{2.5cm}{picture.jpg}
\personalinfo{%
  % Not all of these are required!
  % You can add your own with \printinfo{symbol}{detail}
  \email{l.schlatterer@gmail.com}
  \phone{+41 76 712 28 82}
  % \mailaddress{Address, Street, 00000 County}
  \location{Rosenstrasse 42, 3800, Interlaken, Schweiz}
  % \homepage{www.pudr.com}
  % \twitter{MichalBryxi}
  \linkedin{linkedin.com/in/TBD}
  % \printinfo{\faGitlab}{bar}
  % \github{github.com/mmayer} % I'm just making this up though.
%   \orcid{orcid.org/0000-0000-0000-0000} % Obviously making this up too. If you want to use this field (and also other academicons symbols), add "academicons" option to \documentclass{altacv}
}

%% Make the header extend all the way to the right, if you want.
\begin{fullwidth}
  \makecvheader
\end{fullwidth}


%% Depending on your tastes, you may want to make fonts of itemize environments slightly smaller
\AtBeginEnvironment{itemize}{\small}

%% Provide the file name containing the sidebar contents as an optional parameter to \cvsection.
%% You can always just use \marginpar{...} if you do
%% not need to align the top of the contents to any
%% \cvsection title in the "main" bar.
\cvsection[page1sidebar]{Berufserfahrung}

  \cvevent{Kursleiterin}{Volkshochschule plus}{01/2020 -- heute}{Bern}
    % \begin{itemize}
      % \item TBD
      % \item TBD
    % \end{itemize}
  \divider

  \cvevent{Mitarbeiterin Berufliche Integration und Job Coaching}{SEEBURG}{11/2020 -- heute}{Interlaken}
    % \begin{itemize}
      % \item TBD
      % \item TBD
    % \end{itemize}
  \divider

  \cvevent{Stellvertretende Wohnverantwortung}{Wohnbereich Schlössli, SEEBURG}{01/2020 -- 10/2020}{Interlaken}
    % \begin{itemize}
      % \item TBD
      % \item TBD
    % \end{itemize}
  \divider

  \cvevent{Mitarbeiterin Betreuung}{Wohnbereich Schlössli, SEEBURG}{08/2019 -- 12/2019}{Interlaken}
    % \begin{itemize}
      % \item TBD
      % \item TBD
    % \end{itemize}
  \divider

  \cvevent{Stellvertretende Teamleitung Kindergruppe 4}{Stiftung Sunneschyn}{05/2019 -- 07/2019}{Meiringen}
    % \begin{itemize}
      % \item{TBD}
    % \end{itemize}
  \divider

  \cvevent{Bewerbungsphase}{}{02/2019 -- 04/2019}{}
  % \begin{itemize}
    % \item{TBD}
  % \end{itemize}
  \divider

  \cvevent{Arbeitsassistentin}{Hamburger Arbeitsassistenz gGmbH}{10/2016 -- 11/2018}{Hamburg}
  \begin{itemize}
    \item{Mitarbeit im Integrationsfachdienst für Menschen mit Lernschwierigkeiten}
    \item{Unterstützung im Bereich Profilerstellung, Bewerbungsphase, Einarbeitung, Praktika, Arbeitsplatz}
    \item {Jobcoaching in Betrieben des allgemeinen Arbeitsmarktes}
    \item {Erstellen von Anträgen, Berichten und Zertifikaten}
    \item {Systemische Arbeit mit dem persönlichen Umfeld der Klient\_innen und aktive Vernetzung mit anderen Trägern}
    \item {Akquisition von Betrieben des allgemeinen Arbeitsmarktes}
    \item {Mitarbeit im Erasmus+ Projekt "Valueable"}
  \end{itemize}
  \divider

  \cvevent{Mitarbeiterin}{Freizeiteinrichtung Lebenshilfe}{07/2008 -- 08/2016}{Ludwigsburg}
  \begin{itemize}
    \item{Initiierung und Leitung eines Autonomiekurses für Jugendliche mit Lernschwierigkeiten nach italienischen Vorbild}
    \item {Schülerbetreuungen und Individuelle Einzelbetreuungen für Schüler\_innen mit Lernschwierigkeiten}
    \item {Wöchentliche Kurse für Schüler\_innen und Erwachsene mit Lernschwierigkeiten}
    \item {Ferien für Schüler\_innen und Erwachsene mit Lernschwierigkeiten}
  \end{itemize}
  \divider

% \clearpage

% \begin{fullwidth}
\cvsection[page2sidebar]{Seminare, Fachtagungen und Fortbildungen}
  \cvachievement{\faMicrophone}{09/2022 - heute}{Weiterbildung zur Moderatorin “Persönliche Zukunftsplanung”}
  \divider

  \cvachievement{\faRocket}{05/2018}{Fortbildung „Intervision- Kollegiale Beratung“}
  \divider

  \cvachievement{\faTree}{03/2018}{Fortbildung „Einführung in die Leichte Sprache“}
  \divider

\cvsection{Hobbies}

% Adapted from @Jake's answer from http://tex.stackexchange.com/a/82729/226
% \wheelchart{outer radius}{inner radius}{
% comma-separated list of value/text width/color/detail}
% Some ad-hoc tweaking to adjust the labels so that they don't overlap
\wheelchart{1.5cm}{0.5cm}{%
  50/10em/accent!90/Gleitschirmfliegen,
  20/9em/accent!80/Draussen sein,
  10/15em/accent!50/Neues lernen,
  10/13em/accent!40/Backen,
  10/8em/accent!30/Lesen
}
  % \end{fullwidth}
% \cvsection[page2sidebar]{Publications}

% \nocite{*}

% \printbibliography[heading=pubtype,title={\printinfo{\faBook}{Books}},type=book]

% \divider

% \printbibliography[heading=pubtype,title={\printinfo{\faFileTextO}{Journal Articles}}, type=article]

% \divider

% \printbibliography[heading=pubtype,title={\printinfo{\faGroup}{Conference Proceedings}},type=inproceedings]

%% If the NEXT page doesn't start with a \cvsection but you'd
%% still like to add a sidebar, then use this command on THIS
%% page to add it. The optional argument lets you pull up the
%% sidebar a bit so that it looks aligned with the top of the
%% main column.
% \addnextpagesidebar[-1ex]{page3sidebar}

\end{document}
